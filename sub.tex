\documentclass[10pt,english]{article}
\raggedbottom
\sloppy
\usepackage{graphicx} % Required for inserting images
\usepackage{amsmath}
\usepackage{array}     % Required for p{} columns in multicolumn with xcolor
\usepackage{authblk}
\usepackage{algorithm}
\usepackage{algpseudocode}
\usepackage[english]{babel}
\usepackage{setspace}
\usepackage{times}
\usepackage{indentfirst}
\usepackage{titlepic}
\usepackage{enumitem}
\usepackage[utf8]{inputenc}
\usepackage[T5]{fontenc}
\usepackage{multirow}
\usepackage{booktabs}
\usepackage{minted}
\usepackage{hyperref}
\usepackage{charter}
\usepackage{float}
\usepackage{subfig}  % For subfigure support
\usepackage[table,xcdraw]{xcolor}
\makeatletter
\@ifundefined{insert@pcolumn}{%
  \let\insert@pcolumn\insert@column}{}
\makeatother
\usepackage[letterpaper,top=2cm,bottom=2cm,left=2.75cm,right=2.75cm,marginparwidth=1.75cm]{geometry}

\graphicspath{ {./images/} }
\setcounter{tocdepth}{3}

% --- Global spacing shrink by 20% ---
% Put in the preamble, after \documentclass
\usepackage{calc}      % allows 0.4\length arithmetic
\usepackage{setspace}  % for line spacing
\usepackage{enumitem}  % for list spacing
\usepackage{titlesec}  % for section spacing (optional but helpful)

% 1) Line spacing (baseline)
% If you previously used \onehalfspacing or \setstretch{1.2}, reduce by 20% of that:
% Example: from 1.2 -> 0.96
\setstretch{0.4} % 0.4 × the default line spacing (tight!)
% If you were already using \setstretch{s}, replace with \setstretch{0.4*s}.

% 2) Paragraph spacing/indent
\setlength{\parskip}{.4\parskip}
\setlength{\parindent}{.4\parindent}

% 3) Displayed math spacing
\setlength{\abovedisplayskip}{.4\abovedisplayskip}
\setlength{\belowdisplayskip}{.4\belowdisplayskip}
\setlength{\abovedisplayshortskip}{.4\abovedisplayshortskip}
\setlength{\belowdisplayshortskip}{.4\belowdisplayshortskip}

% 4) Floats (figures/tables) spacing
\setlength{\textfloatsep}{.4\textfloatsep}
\setlength{\floatsep}{.4\floatsep}
\setlength{\intextsep}{.4\intextsep}
\setlength{\dbltextfloatsep}{.4\dbltextfloatsep}
\setlength{\dblfloatsep}{.4\dblfloatsep}

% 5) Lists spacing (itemize/enumerate/description)
\setlist{itemsep=.2\itemsep, topsep=.2\topsep, parsep=.2\parsep, partopsep=.2\partopsep}

% 6) Section heading spacing (before/after)
% Adjust all levels uniformly; tweak if needed.
\titlespacing*{\section}{0pt}{.4\baselineskip}{.4\baselineskip}
\titlespacing*{\subsection}{0pt}{.4\baselineskip}{.4\baselineskip}
\titlespacing*{\subsubsection}{0pt}{.4\baselineskip}{.4\baselineskip}

% (Optional) Tighten captions slightly (requires caption package)
% \usepackage[skip=.4\baselineskip]{caption}


\begin{document}
 
\begin{titlepage}
\begin{center}

\textsc{\large \textcolor{black}{\textbf{HANOI UNIVERSITY OF SCIENCE AND TECHNOLOGY }}}\\[0.5cm]

{\huge \Large \uppercase{School of Information Communication Technology} \\[1cm] }

\includegraphics[scale=0.13]{logo-soict-hust-1.png}\\[1cm]
\textsc{\Large \textbf{Project report:}}\\[0.4cm]	

\rule{\linewidth}{0.7mm} \\[0.4cm]
{ \huge \bfseries\color{black!70!black} Laptop Store Management System \\[0.4cm] }
\rule{\linewidth}{0.7mm} \\[1cm]

\color{black!80!black}{\large \textbf{\textit{Supervised by}:}}\\[0.4cm]

\begin{tabular}{l}
\large  Ph.D. Nguyen Duc Anh  \\[1cm]
\end{tabular}

\color{black!80!black}{\large \textbf{\textit{Presented by}:}}\\[0.4cm]
\color{black}
\begin{tabular}{l}

\large Au Trung Phong - 20225455 \\[0.4cm]
\large Nguyen Nam Khanh - 20225448 \\[0.4cm]
\large Vu Duc Minh - 20225514 \\[0.4cm]
\large Tran Sy Minh Quan - 20225521 \\[0.4cm]
\large Dao Vu Tien Dat - 20225479 \\[0.4cm]
\large Tran Huu Dao - 20220061 \\[0.4cm]

\end{tabular}

\vfill

{\large \color{black!80!black}{\textbf{Hanoi - Vietnam}}\\[0.2cm] \color{blue!90!black}2025}

\end{center}
\end{titlepage}
\setstretch{1.5}


\tableofcontents

\section{Introduction}

\subsection{Problem Definition}
\subsection{Problem Solving}


\section{Planning}

\subsection{Motivation}
\subsection{Initial System Request}
\subsubsection{Business Need}
\subsubsection{Business Requirements}
\subsubsection{Business Value}
\subsubsection{Special Issues or Constraints}


\subsection{Feasibility Analysis}

\subsubsection{Technical Feasibility}
\subsubsection{Operational Feasibility}
\subsubsection{Economic Feasibility}
\subsubsection{Legal \& Regulatory Feasibility}
\subsubsection{Market Feasibility}

\subsection{Initial Project Plan}

\subsubsection{Project Scope Estimation}

\subsubsection{Staffing Plan}

\subsubsection{CASE Tools}

\clearpage

\section{Analysis}
\subsection{General Use Cases Diagram}

\begin{figure}[H]
    \centering
    \includegraphics[width=1.0\textwidth]{overall_uc.drawio.png}
    \caption{Overall Use Case of Laptop Shop Management System}
    \label{fig:backend_architecture}
\end{figure}

The general use case diagram of the \textbf{Laptop Store Management System} illustrates the main interactions between users and the system. It represents the overall system functionality and identifies the actors who participate in the system’s operations.

\paragraph{Actors:}
\begin{itemize}
    \item \textbf{Customer:} The main user who can browse products, view laptop details, register and log in, add laptops to the cart, place orders, make payments, track orders, request refunds, and submit reviews or ratings.
    \item \textbf{Shop Owner (Admin):} The administrator who manages the overall system operations, including inventory management, order processing, and refund handling.
    \item \textbf{E-Bank Payment Gateway:} An external system integrated for processing online transactions through e-banking.
    \item \textbf{User:} A general actor representing both registered and unregistered customers interacting with the system.
\end{itemize}

\paragraph{System Overview:}
The \textbf{Laptop Store Management System} allows customers to explore available laptops, view detailed specifications, and make purchases through a streamlined online process. Registered users can manage their shopping carts, complete payments via multiple methods (including e-banking or cash on delivery), and track or request refunds for their orders. The Shop Owner manages product inventory, reviews customer refund tickets, and updates the status of customer orders.

\paragraph{Diagram Description:}
The general use case diagram demonstrates the main functionalities of the system and their relationships with each actor. The diagram includes the following major interactions:
\begin{itemize}
    \item Customers can \textit{Register} and \textit{Login} to access their accounts.
    \item Once logged in, they can \textit{Explore Products}, \textit{Search Laptop}, and \textit{View Laptop Detail}.
    \item From the laptop details, they can \textit{Filter Laptop}, \textit{Add Laptop to Cart}, \textit{Submit Review \& Ratings}, or \textit{Request Refund}.
    \item Customers can manage their shopping cart through actions such as \textit{View Cart}, \textit{Delete Laptop from Cart}, and finally \textit{Place Order}.
    \item The \textit{Make Payment} use case includes two options: \textit{Pay on Delivery} and \textit{Pay via E-banking}, where the latter interacts directly with the \textbf{E-Bank Payment Gateway}.
    \item After completing orders, customers can \textit{Track Order} and, if necessary, initiate a refund request.
    \item The \textbf{Shop Owner (Admin)} can \textit{View Dashboard}, \textit{Add/Edit/Delete Laptops from Inventory}, \textit{Change Order’s Status}, and \textit{Solve Refund Tickets}.
\end{itemize}

\paragraph{Summary:}
This use case diagram provides an overview of how the system facilitates seamless interaction between customers, administrators, and external services. It defines the scope of user operations and forms the foundation for detailed use case descriptions presented in Section~3.2.

\subsection{Use Case Details}

This section presents all use case metadata cards for the Laptop Store Management System.
Each use case diagram illustrates the interaction flow between the actors and the system.
They are grouped by actor roles, following the same layout as the G8 System report.

% =====================================================
\subsubsection{Customer’s Use Case Details}
% =====================================================

\begin{figure}[H]
    \centering
    \includegraphics[width=\textwidth, height=0.85\textheight, keepaspectratio]{usecase_metadata_card/uc_0_register.drawio.png}
    \caption{U0 – Register}
\end{figure}

\begin{figure}[H]
    \centering
    \includegraphics[width=0.95\textwidth]{usecase_metadata_card/uc_1_login.drawio.png}
    \caption{U1 – Login}
\end{figure}

\begin{figure}[H]
    \centering
    \includegraphics[width=0.95\textwidth]{usecase_metadata_card/uc_2_explore_products.drawio.png}
    \caption{U2 – Explore Products}
\end{figure}

\begin{figure}[H]
    \centering
    \includegraphics[width=0.95\textwidth]{usecase_metadata_card/uc_3_search_laptop.drawio.png}
    \caption{U3 – Search Laptop}
\end{figure}

\begin{figure}[H]
    \centering
    \includegraphics[width=0.95\textwidth]{usecase_metadata_card/uc_4_filter_laptop.drawio.png}
    \caption{U4 – Filter Laptop}
\end{figure}

\begin{figure}[H]
    \centering
    \includegraphics[width=0.95\textwidth]{usecase_metadata_card/uc_5_view_details.drawio.png}
    \caption{U5 – View Laptop Details}
\end{figure}

\begin{figure}[H]
    \centering
    \includegraphics[width=0.95\textwidth]{usecase_metadata_card/uc_6_add_laptop_to_cart.drawio.png}
    \caption{U6 – Add Laptop to Cart}
\end{figure}

\begin{figure}[H]
    \centering
    \includegraphics[width=0.95\textwidth]{usecase_metadata_card/uc_7_view_cart.drawio.png}
    \caption{U7 – View Cart}
\end{figure}

\begin{figure}[H]
    \centering
    \includegraphics[width=0.95\textwidth]{usecase_metadata_card/uc_8_delete_laptop_from_cart.drawio.png}
    \caption{U8 – Delete Laptop from Cart}
\end{figure}

\begin{figure}[H]
    \centering
    \includegraphics[width=0.95\textwidth]{usecase_metadata_card/uc_9_place_order.drawio.png}
    \caption{U9 – Place Order}
\end{figure}

\begin{figure}[H]
    \centering
    \includegraphics[width=0.95\textwidth]{usecase_metadata_card/uc_10_make_payment.drawio.png}
    \caption{U10 – Make Payment}
\end{figure}

\begin{figure}[H]
    \centering
    \includegraphics[width=0.95\textwidth]{usecase_metadata_card/uc_11_pay_on_delivery.drawio.png}
    \caption{U11 – Pay on Delivery}
\end{figure}

\begin{figure}[H]
    \centering
    \includegraphics[width=0.95\textwidth]{usecase_metadata_card/uc_12_pay_via_e-banking.drawio.png}
    \caption{U12 – Pay via E-banking}
\end{figure}

\begin{figure}[H]
    \centering
    \includegraphics[width=0.95\textwidth]{usecase_metadata_card/uc_13_track_order.drawio.png}
    \caption{U13 – Track Order}
\end{figure}

\begin{figure}[H]
    \centering
    \includegraphics[width=0.95\textwidth]{usecase_metadata_card/uc_14_request_refund.drawio.png}
    \caption{U14 – Request Refund}
\end{figure}

\begin{figure}[H]
    \centering
    \includegraphics[width=0.95\textwidth]{usecase_metadata_card/uc_15_submit_reviews_and_ratings.drawio.png}
    \caption{U15 – Submit Reviews and Ratings}
\end{figure}

% =====================================================
\subsubsection{Admin’s Use Case Details}
% =====================================================

\begin{figure}[H]
    \centering
    \includegraphics[width=0.95\textwidth]{usecase_metadata_card/uc_16_view_dashboard.drawio.png}
    \caption{U16 – View Dashboard}
\end{figure}

\begin{figure}[H]
    \centering
    \includegraphics[width=0.95\textwidth]{usecase_metadata_card/uc_17_add_laptop_to_inventory.drawio.png}
    \caption{U17 – Add Laptop to Inventory}
\end{figure}

\begin{figure}[H]
    \centering
    \includegraphics[width=0.95\textwidth]{usecase_metadata_card/uc_18_edit_laptop_from_inventory.drawio.png}
    \caption{U18 – Edit Laptop from Inventory}
\end{figure}

\begin{figure}[H]
    \centering
    \includegraphics[width=0.95\textwidth]{usecase_metadata_card/uc_19_delete_laptop_from_inventory.drawio.png}
    \caption{U19 – Delete Laptop from Inventory}
\end{figure}

\begin{figure}[H]
    \centering
    \includegraphics[width=0.95\textwidth]{usecase_metadata_card/uc_20_change_order’s_status.drawio.png}
    \caption{U20 – Update Order State}
\end{figure}

\begin{figure}[H]
    \centering
    \includegraphics[width=0.95\textwidth]{usecase_metadata_card/uc_21_solve_refund_tickets.drawio.png}
    \caption{U21 – Solve Refund Tickets}
\end{figure}

% =====================================================
\subsection{Activity Diagram}
% =====================================================

% ------------------- CUSTOMER -------------------
\subsubsection{Customer’s Activity Diagram}

\begin{figure}[H]
    \centering
    \includegraphics[width=\textwidth, height=0.85\textheight, keepaspectratio]{Activity_Diagram/C1_register.drawio.png}
    \caption{U0 – Register}
\end{figure}

\begin{figure}[H]
    \centering
    \includegraphics[width=\textwidth, height=0.85\textheight, keepaspectratio]{Activity_Diagram/C2_login.drawio.png}
    \caption{U1 – Login}
\end{figure}

\begin{figure}[H]
    \centering
    \includegraphics[width=\textwidth, height=0.85\textheight, keepaspectratio]{Activity_Diagram/UC_2.drawio.png}
    \caption{U2 – Explore Products}
\end{figure}

\begin{figure}[H]
    \centering
    \includegraphics[width=\textwidth, height=0.85\textheight, keepaspectratio]{Activity_Diagram/UC_3.drawio.png}
    \caption{U3 – Search Laptop}
\end{figure}

\begin{figure}[H]
    \centering
    \includegraphics[width=\textwidth, height=0.85\textheight, keepaspectratio]{Activity_Diagram/UC4.drawio.png}
    \caption{U4 – Filter Laptop}
\end{figure}

\begin{figure}[H]
    \centering
    \includegraphics[width=\textwidth, height=0.85\textheight, keepaspectratio]{Activity_Diagram/UC5.drawio.png}
    \caption{U5 – View Laptop Details}
\end{figure}

\begin{figure}[H]
    \centering
    \includegraphics[width=\textwidth, height=0.85\textheight, keepaspectratio]{Activity_Diagram/C3_add_laptop.drawio.png}
    \caption{U6 – Add Laptop to Cart}
\end{figure}

\begin{figure}[H]
    \centering
    \includegraphics[width=\textwidth, height=0.85\textheight, keepaspectratio]{Activity_Diagram/UC7.drawio.png}
    \caption{U7 – View Cart}
\end{figure}

\begin{figure}[H]
    \centering
    \includegraphics[width=\textwidth, height=0.85\textheight, keepaspectratio]{Activity_Diagram/C4_delete_cart.drawio.png}
    \caption{U8 – Delete Laptop from Cart}
\end{figure}

\begin{figure}[H]
    \centering
    \includegraphics[width=\textwidth, height=0.85\textheight, keepaspectratio]{Activity_Diagram/C5.drawio.png}
    \caption{U9 – Place Order}
\end{figure}

\begin{figure}[H]
    \centering
    \includegraphics[width=\textwidth, height=0.85\textheight, keepaspectratio]{Activity_Diagram/C6.drawio.png}
    \caption{U10 – Make Payment}
\end{figure}

\begin{figure}[H]
    \centering
    \includegraphics[width=\textwidth, height=0.85\textheight, keepaspectratio]{Activity_Diagram/C7.drawio.png}
    \caption{U13 – Track Order}
\end{figure}

\begin{figure}[H]
    \centering
    \includegraphics[width=\textwidth, height=0.85\textheight, keepaspectratio]{Activity_Diagram/C8.drawio.png}
    \caption{U14 – Request Refund}
\end{figure}

\begin{figure}[H]
    \centering
    \includegraphics[width=\textwidth, height=0.85\textheight, keepaspectratio]{Activity_Diagram/U15.drawio.png}
    \caption{U15 – Submit Reviews and Ratings}
\end{figure}

\clearpage

% ------------------- ADMIN -------------------
\subsubsection{Admin’s Activity Diagram}

\begin{figure}[H]
    \centering
    \includegraphics[width=\textwidth, height=0.85\textheight, keepaspectratio]{Activity_Diagram/A6.drawio.png}
    \caption{U16 –  View Dashboard}
\end{figure}

\begin{figure}[H]
    \centering
    \includegraphics[width=\textwidth, height=0.85\textheight, keepaspectratio]{Activity_Diagram/A1.drawio.png}
    \caption{U17 – Add Laptop to Inventory}
\end{figure}

\begin{figure}[H]
    \centering
    \includegraphics[width=\textwidth, height=0.85\textheight, keepaspectratio]{Activity_Diagram/A2.drawio.png}
    \caption{U18 – Edit Laptop from Inventory}
\end{figure}

\begin{figure}[H]
    \centering
    \includegraphics[width=\textwidth, height=0.85\textheight, keepaspectratio]{Activity_Diagram/A3.drawio.png}
    \caption{U19 – Delete Laptop from Inventory}
\end{figure}

\begin{figure}[H]
    \centering
    \includegraphics[width=\textwidth, height=0.85\textheight, keepaspectratio]{Activity_Diagram/A4.drawio.png}
    \caption{U20 – Update Order State}
\end{figure}

\begin{figure}[H]
    \centering
    \includegraphics[width=\textwidth, height=0.85\textheight, keepaspectratio]{Activity_Diagram/A5.drawio.png}
    \caption{U21 – Solve Refund Ticket}
\end{figure}


\subsection{Sequence Diagram}
% \subsubsection{Customer’s Use Case}
% \subsubsection{Admin’s Use Case Details}
% ============================
%   SEQUENCE DIAGRAM FIGURES
% ============================

\begin{figure}[H]
    \centering
    \includegraphics[width=\textwidth, height=0.85\textheight, keepaspectratio]{Sequence_Diagram/UC_0_Register.png}
    \caption{UC 0 – Register}
\end{figure}

\begin{figure}[H]
    \centering
    \includegraphics[width=\textwidth, height=0.85\textheight, keepaspectratio]{Sequence_Diagram/UC_1_Login.png}
    \caption{UC 1 – Login}
\end{figure}

\begin{figure}[H]
    \centering
    \includegraphics[width=\textwidth, height=0.85\textheight, keepaspectratio]{Sequence_Diagram/UC_2_ExploreProducts.png}
    \caption{UC 2 – Explore Products}
\end{figure}

\begin{figure}[H]
    \centering
    \includegraphics[width=\textwidth, height=0.85\textheight, keepaspectratio]{Sequence_Diagram/UC_3_SearchLaptop-Trang-1.drawio.png}
    \caption{UC 3 – Search Laptop}
\end{figure}

\begin{figure}[H]
    \centering
    \includegraphics[width=\textwidth, height=0.85\textheight, keepaspectratio]{Sequence_Diagram/UC_4_FilterLaptop.drawio.png}
    \caption{UC 4 – Filter Laptop}
\end{figure}

\begin{figure}[H]
    \centering
    \includegraphics[width=\textwidth, height=0.85\textheight, keepaspectratio]{Sequence_Diagram/UC_5_ViewDetail.png}
    \caption{UC 5 – View Detail}
\end{figure}

\begin{figure}[H]
    \centering
    \includegraphics[width=\textwidth, height=0.85\textheight, keepaspectratio]{Sequence_Diagram/UC_6_AddLaptopToCart.png}
    \caption{UC 6 – Add Laptop to Cart}
\end{figure}

\begin{figure}[H]
    \centering
    \includegraphics[width=\textwidth, height=0.85\textheight, keepaspectratio]{Sequence_Diagram/UC_7_ViewCart.drawio.png}
    \caption{UC 7 – View Cart}
\end{figure}

\begin{figure}[H]
    \centering
    \includegraphics[width=\textwidth, height=0.85\textheight, keepaspectratio]{Sequence_Diagram/UC_8_DeleteLaptopFromCart.png}
    \caption{UC 8 – Delete Laptop from Cart}
\end{figure}

\begin{figure}[H]
    \centering
    \includegraphics[width=\textwidth, height=0.85\textheight, keepaspectratio]{Sequence_Diagram/UC_9_PlaceOrder.png}
    \caption{UC 9 – Place Order}
\end{figure}

\begin{figure}[H]
    \centering
    \includegraphics[width=\textwidth, height=0.85\textheight, keepaspectratio]{Sequence_Diagram/UC_10_MakePayment.drawio.png}
    \caption{UC 10 – Make Payment}
\end{figure}

\begin{figure}[H]
    \centering
    \includegraphics[width=\textwidth, height=0.85\textheight, keepaspectratio]{Sequence_Diagram/UC_13_TrackOrder.png}
    \caption{UC 13 – Track Order}
\end{figure}

\begin{figure}[H]
    \centering
    \includegraphics[width=\textwidth, height=0.85\textheight, keepaspectratio]{Sequence_Diagram/UC_14_RequestRefund.drawio.png}
    \caption{UC 14 – Request Refund}
\end{figure}

\begin{figure}[H]
    \centering
    \includegraphics[width=\textwidth, height=0.85\textheight, keepaspectratio]{Sequence_Diagram/UC_15_SubmitReviewAndRating.drawio.png}
    \caption{UC 15 – Submit Review and Rating}
\end{figure}

\begin{figure}[H]
    \centering
    \includegraphics[width=\textwidth, height=0.85\textheight, keepaspectratio]{Sequence_Diagram/UC_16_ViewDashboard.drawio.png}
    \caption{UC 16 – View Dashboard}
\end{figure}

\begin{figure}[H]
    \centering
    \includegraphics[width=\textwidth, height=0.85\textheight, keepaspectratio]{Sequence_Diagram/UC_17_AddLaptopToInventory.drawio.png}
    \caption{UC 17 – Add Laptop to Inventory}
\end{figure}

\begin{figure}[H]
    \centering
    \includegraphics[width=\textwidth, height=0.85\textheight, keepaspectratio]{Sequence_Diagram/UC_18_EditLaptopFromInventory.drawio.png}
    \caption{UC 18 – Edit Laptop from Inventory}
\end{figure}

\begin{figure}[H]
    \centering
    \includegraphics[width=\textwidth, height=0.85\textheight, keepaspectratio]{Sequence_Diagram/UC_19_DeleteLaptopFromInventory.drawio.png}
    \caption{UC 19 – Delete Laptop from Inventory}
\end{figure}

\begin{figure}[H]
    \centering
    \includegraphics[width=\textwidth, height=0.85\textheight, keepaspectratio]{Sequence_Diagram/UC_20_ChangeOrder_sStatus.drawio.png}
    \caption{UC 20 – Change Order State}
\end{figure}

\begin{figure}[H]
    \centering
    \includegraphics[width=\textwidth, height=0.85\textheight, keepaspectratio]{Sequence_Diagram/UC_21_SolveRefundTicket.drawio.png}
    \caption{UC 21 – Solve Refund Ticket}
\end{figure}


\subsection{Non-functional Requirements}

\section{Design}

% =====================================================
\subsection{Architecture Design}
% =====================================================

\subsubsection{Package Diagram}
The Laptop Store Management System is architected using the \textbf{Model--View--Controller (MVC)} design pattern, which ensures a clear separation of concerns, maintainability, and scalability. This architecture divides the application into three interconnected components:

\begin{itemize}
    \item \textbf{Model Layer}: This layer represents the core business logic and data state of the application. It encapsulates entities specific to the domain, such as \texttt{Laptop} (product details, stock), \texttt{User} (customer and admin profiles), \texttt{Order} (transaction history), and \texttt{Cart} (temporary shopping items). The Model is responsible for data integrity and business rules, independent of the user interface.
    
    \item \textbf{View Layer}: The View layer is responsible for the presentation of data to the user. In our system, this includes the graphical user interfaces (GUI) such as the \textit{Product Catalog} for browsing laptops, the \textit{Shopping Cart} screen, the \textit{Checkout} interface, and the \textit{Admin Dashboard} for inventory management. The View observes the Model and updates the display when the state changes.
    
    \item \textbf{Controller Layer}: The Controller acts as an intermediary between the View and the Model. It handles user interactions, such as adding a laptop to the cart or processing a payment. Controllers like \texttt{LaptopController} and \texttt{OrderController} receive input from the View, validate it, and invoke the appropriate business logic in the Model.
\end{itemize}

The dependency direction is strictly \textbf{View $\rightarrow$ Controller $\rightarrow$ Model}, ensuring that the business logic remains decoupled from the user interface. This structure allows for easier testing and future enhancements, such as adding new payment methods or modifying the UI without affecting the core system logic.

% If you have an overall package diagram PNG, insert it here.
\begin{figure}[H]
    \centering
    \includegraphics[width=0.85\textwidth]{Package_Diagram/PackageDiagram.drawio.png}
    \caption{MVC Package Diagram}
    \label{fig:pkg_diagram}
\end{figure}

\subsubsection{Detailed Package Diagram}

\subsubsection{Model Package}
\begin{figure}[H]
    \centering
    \includegraphics[width=0.95\textwidth]{Package_Diagram/ModelPackage.drawio.png}
    \caption{Detailed Model Package Diagram}
    \label{fig:model_package}
\end{figure}

\subsubsection{View Package}
\begin{figure}[H]
    \centering
    \includegraphics[width=0.95\textwidth]{Package_Diagram/ViewPackage.drawio.png}
    \caption{Detailed View Package Diagram}
    \label{fig:view_package}
\end{figure}

\subsubsection{Controller Package}
\begin{figure}[H]
    \centering
    \includegraphics[width=0.95\textwidth]{Package_Diagram/ControllerPackage.drawio.png}
    \caption{Detailed Controller Package Diagram}
    \label{fig:controller_package}
\end{figure}

\subsection{Class Specification Design}

\subsubsection{Model Package}

% --- M_User ---
\subsubsection{Class M\_User}
\noindent\textbf{Description:} Represents a registered user within the system, encapsulating all necessary authentication credentials (email, password hash) and role-based access control data (Admin or Customer). This class serves as the foundation for user identity management and security.

\noindent\textbf{Attributes:}
\begin{center}
\begin{tabular}{|p{3cm}|p{2.5cm}|p{2.5cm}|p{6cm}|}
\hline
\textbf{Attribute} & \textbf{Datatype} & \textbf{Accessibility} & \textbf{Description} \\ \hline
userId & UUID & Public & Unique identifier for the user. \\ \hline
email & String & Public & User's email address (login credential). \\ \hline
role & String & Public & The role of the user (e.g., Admin, Customer). \\ \hline
\end{tabular}
\end{center}

\noindent\textbf{Methods:}
\begin{center}
\begin{tabular}{|p{3.5cm}|p{2cm}|p{2cm}|p{2cm}|p{4cm}|}
\hline
\textbf{Method} & \textbf{Input} & \textbf{Output} & \textbf{Access} & \textbf{Description} \\ \hline
verifyPassword & - & - & Public & Verifies the user's password. \\ \hline
deactivate & - & - & Public & Deactivates the user account. \\ \hline
\end{tabular}
\end{center}

% --- M_Laptop ---
\subsubsection{Class M\_Laptop}
\noindent\textbf{Description:} Represents a specific laptop product available in the store's inventory. It holds comprehensive product details including technical specifications, brand information, current pricing, and real-time stock availability to prevent overselling.

\noindent\textbf{Attributes:}
\begin{center}
\begin{tabular}{|p{3cm}|p{2.5cm}|p{2.5cm}|p{6cm}|}
\hline
\textbf{Attribute} & \textbf{Datatype} & \textbf{Accessibility} & \textbf{Description} \\ \hline
laptopId & UUID & Public & Unique identifier for the laptop. \\ \hline
brand & String & Public & The manufacturer brand. \\ \hline
price & Decimal & Public & Selling price of the laptop. \\ \hline
stockQty & int & Public & Quantity currently in stock. \\ \hline
\end{tabular}
\end{center}

\noindent\textbf{Methods:}
\begin{center}
\begin{tabular}{|p{3.5cm}|p{2cm}|p{2cm}|p{2cm}|p{4cm}|}
\hline
\textbf{Method} & \textbf{Input} & \textbf{Output} & \textbf{Access} & \textbf{Description} \\ \hline
isInStock & - & - & Public & Checks if stock is available. \\ \hline
adjustStock & - & - & Public & Modifies the stock quantity. \\ \hline
\end{tabular}
\end{center}

% --- M_Review ---
\subsubsection{Class M\_Review}
\noindent\textbf{Description:} Represents a customer's feedback on a purchased product. It stores the numerical rating and textual comment provided by the user, linking the sentiment to a specific laptop and customer account.

\noindent\textbf{Attributes:}
\begin{center}
\begin{tabular}{|p{3cm}|p{2.5cm}|p{2.5cm}|p{6cm}|}
\hline
\textbf{Attribute} & \textbf{Datatype} & \textbf{Accessibility} & \textbf{Description} \\ \hline
reviewId & UUID & Public & Unique identifier for the review. \\ \hline
rating & int & Public & Numerical rating score. \\ \hline
comment & String & Public & Text content of the review. \\ \hline
\end{tabular}
\end{center}

\noindent\textbf{Methods:}
\begin{center}
\begin{tabular}{|p{3.5cm}|p{2cm}|p{2cm}|p{2cm}|p{4cm}|}
\hline
\textbf{Method} & \textbf{Input} & \textbf{Output} & \textbf{Access} & \textbf{Description} \\ \hline
validate & - & - & Public & Validates review content. \\ \hline
\end{tabular}
\end{center}

% --- M_Cart ---
\subsubsection{Class M\_Cart}
\noindent\textbf{Description:} Represents a temporary shopping session for a user. It acts as a container for selected items before purchase, maintaining the running total cost and managing the lifecycle of the shopping experience.

\noindent\textbf{Attributes:}
\begin{center}
\begin{tabular}{|p{3cm}|p{2.5cm}|p{2.5cm}|p{6cm}|}
\hline
\textbf{Attribute} & \textbf{Datatype} & \textbf{Accessibility} & \textbf{Description} \\ \hline
cartId & UUID & Public & Unique identifier for the cart. \\ \hline
totalAmount & Decimal & Public & Total cost of items in cart. \\ \hline
\end{tabular}
\end{center}

\noindent\textbf{Methods:}
\begin{center}
\begin{tabular}{|p{3.5cm}|p{2cm}|p{2cm}|p{2cm}|p{4cm}|}
\hline
\textbf{Method} & \textbf{Input} & \textbf{Output} & \textbf{Access} & \textbf{Description} \\ \hline
addItem & - & - & Public & Adds a product to the cart. \\ \hline
recalculateTotal & - & - & Public & Updates the total amount. \\ \hline
\end{tabular}
\end{center}

% --- M_CartItem ---
\subsubsection{Class M\_CartItem}
\noindent\textbf{Description:} Represents a single line item within a shopping cart, linking a specific laptop product to a desired quantity. It calculates its own subtotal based on the current price and selected quantity.

\noindent\textbf{Attributes:}
\begin{center}
\begin{tabular}{|p{3cm}|p{2.5cm}|p{2.5cm}|p{6cm}|}
\hline
\textbf{Attribute} & \textbf{Datatype} & \textbf{Accessibility} & \textbf{Description} \\ \hline
itemId & UUID & Public & Unique identifier for the item. \\ \hline
quantity & int & Public & Number of units selected. \\ \hline
subtotal & Decimal & Public & Price multiplied by quantity. \\ \hline
\end{tabular}
\end{center}

% --- M_Order ---
\subsubsection{Class M\_Order}
\noindent\textbf{Description:} Represents a formalized purchase agreement. Once a cart is checked out, it becomes an order. This class tracks the lifecycle of the transaction from creation through processing to completion, including the final total amount and current status.

\noindent\textbf{Attributes:}
\begin{center}
\begin{tabular}{|p{3cm}|p{2.5cm}|p{2.5cm}|p{6cm}|}
\hline
\textbf{Attribute} & \textbf{Datatype} & \textbf{Accessibility} & \textbf{Description} \\ \hline
orderId & UUID & Public & Unique identifier for the order. \\ \hline
status & String & Public & Current order state (e.g., Pending). \\ \hline
totalAmount & Decimal & Public & Final cost of the order. \\ \hline
\end{tabular}
\end{center}

\noindent\textbf{Methods:}
\begin{center}
\begin{tabular}{|p{3.5cm}|p{2cm}|p{2cm}|p{2cm}|p{4cm}|}
\hline
\textbf{Method} & \textbf{Input} & \textbf{Output} & \textbf{Access} & \textbf{Description} \\ \hline
updateStatus & - & - & Public & Updates the order status. \\ \hline
\end{tabular}
\end{center}

% --- M_OrderItem ---
\subsubsection{Class M\_OrderItem}
\noindent\textbf{Description:} Represents a specific product included in a finalized order. Unlike a cart item, this record is immutable once the order is placed, preserving the price and quantity at the time of purchase for historical accuracy.

\noindent\textbf{Attributes:}
\begin{center}
\begin{tabular}{|p{3cm}|p{2.5cm}|p{2.5cm}|p{6cm}|}
\hline
\textbf{Attribute} & \textbf{Datatype} & \textbf{Accessibility} & \textbf{Description} \\ \hline
orderItemId & UUID & Public & Unique identifier for the order item. \\ \hline
quantity & int & Public & Quantity purchased. \\ \hline
subtotal & Decimal & Public & Total price for this line item. \\ \hline
\end{tabular}
\end{center}

% --- M_PaymentTransaction ---
\subsubsection{Class M\_PaymentTransaction}
\noindent\textbf{Description:} Represents the financial record of a payment attempt associated with an order. It tracks the transaction ID from the payment gateway, the amount paid, and the success or failure status of the payment.

\noindent\textbf{Attributes:}
\begin{center}
\begin{tabular}{|p{3cm}|p{2.5cm}|p{2.5cm}|p{6cm}|}
\hline
\textbf{Attribute} & \textbf{Datatype} & \textbf{Accessibility} & \textbf{Description} \\ \hline
transactionId & UUID & Public & Unique identifier for the transaction. \\ \hline
amount & Decimal & Public & Amount paid. \\ \hline
\end{tabular}
\end{center}

\noindent\textbf{Methods:}
\begin{center}
\begin{tabular}{|p{3.5cm}|p{2cm}|p{2cm}|p{2cm}|p{4cm}|}
\hline
\textbf{Method} & \textbf{Input} & \textbf{Output} & \textbf{Access} & \textbf{Description} \\ \hline
markSuccess & - & - & Public & Marks transaction as successful. \\ \hline
\end{tabular}
\end{center}

% --- M_RefundTicket ---
\subsubsection{Class M\_RefundTicket}
\noindent\textbf{Description:} Represents a formal request initiated by a customer to return a product or get a refund. It tracks the reason for the request and its current processing status (e.g., Pending, Approved, Rejected) by an administrator.

\noindent\textbf{Attributes:}
\begin{center}
\begin{tabular}{|p{3cm}|p{2.5cm}|p{2.5cm}|p{6cm}|}
\hline
\textbf{Attribute} & \textbf{Datatype} & \textbf{Accessibility} & \textbf{Description} \\ \hline
ticketId & UUID & Public & Unique identifier for the ticket. \\ \hline
status & String & Public & Status (e.g., Pending, Approved). \\ \hline
\end{tabular}
\end{center}

\noindent\textbf{Methods:}
\begin{center}
\begin{tabular}{|p{3.5cm}|p{2cm}|p{2cm}|p{2cm}|p{4cm}|}
\hline
\textbf{Method} & \textbf{Input} & \textbf{Output} & \textbf{Access} & \textbf{Description} \\ \hline
approve & - & - & Public & Approves the refund. \\ \hline
reject & - & - & Public & Rejects the refund. \\ \hline
\end{tabular}
\end{center}

% --- M_FilterCriteria ---
\subsubsection{Class M\_FilterCriteria}
\noindent\textbf{Description:} A data transfer object (DTO) used to encapsulate the user's search and filter preferences (such as brand selection and price range) when browsing the product catalog. It ensures that search queries are structured and valid.

\noindent\textbf{Attributes:}
\begin{center}
\begin{tabular}{|p{3cm}|p{2.5cm}|p{2.5cm}|p{6cm}|}
\hline
\textbf{Attribute} & \textbf{Datatype} & \textbf{Accessibility} & \textbf{Description} \\ \hline
brand & String & Public & Brand name to filter by. \\ \hline
minPrice & Decimal & Public & Minimum price threshold. \\ \hline
\end{tabular}
\end{center}

\noindent\textbf{Methods:}
\begin{center}
\begin{tabular}{|p{3.5cm}|p{2cm}|p{2cm}|p{2cm}|p{4cm}|}
\hline
\textbf{Method} & \textbf{Input} & \textbf{Output} & \textbf{Access} & \textbf{Description} \\ \hline
isValid & - & - & Public & Validates the filter criteria. \\ \hline
\end{tabular}
\end{center}

% --- M_Metrics ---
\subsubsection{Class M\_Metrics}
\noindent\textbf{Description:} Represents aggregated analytical data used for the administrator's dashboard. It provides high-level insights into business performance, such as total order counts and cumulative revenue over a specific period.

\noindent\textbf{Attributes:}
\begin{center}
\begin{tabular}{|p{3cm}|p{2.5cm}|p{2.5cm}|p{6cm}|}
\hline
\textbf{Attribute} & \textbf{Datatype} & \textbf{Accessibility} & \textbf{Description} \\ \hline
totalOrders & int & Public & Total number of orders placed. \\ \hline
totalRevenue & Decimal & Public & Total revenue generated. \\ \hline
\end{tabular}
\end{center}

\subsubsection{View Package}

% --- V_BaseView ---
\subsubsection{Class V\_BaseView}
\noindent\textbf{Description:} An abstract base class that defines the common structure and behavior for all user interface screens. It provides standard methods for displaying the view, hiding it, and showing error messages, ensuring a consistent look and feel across the application.

\noindent\textbf{Attributes:}
\begin{center}
\begin{tabular}{|p{3cm}|p{2.5cm}|p{2.5cm}|p{6cm}|}
\hline
\textbf{Attribute} & \textbf{Datatype} & \textbf{Accessibility} & \textbf{Description} \\ \hline
title & String & Public & Title of the current view/page. \\ \hline
\end{tabular}
\end{center}

\noindent\textbf{Methods:}
\begin{center}
\begin{tabular}{|p{3.5cm}|p{2cm}|p{2cm}|p{2cm}|p{4cm}|}
\hline
\textbf{Method} & \textbf{Input} & \textbf{Output} & \textbf{Access} & \textbf{Description} \\ \hline
show & - & - & Public & Renders the view. \\ \hline
hide & - & - & Public & Hides the view. \\ \hline
displayError & - & - & Public & Displays an error message. \\ \hline
\end{tabular}
\end{center}

% --- V_HomepageView ---
\subsubsection{Class V\_HomepageView}
\noindent\textbf{Description:} The main entry point of the application for users. It presents a welcoming interface, highlights featured products, and provides navigation options to the full product catalog or the login screen.

\noindent\textbf{Methods:}
\begin{center}
\begin{tabular}{|p{4cm}|p{1.5cm}|p{1.5cm}|p{2cm}|p{4cm}|}
\hline
\textbf{Method} & \textbf{Input} & \textbf{Output} & \textbf{Access} & \textbf{Description} \\ \hline
renderAllProductsPage & - & - & Public & Displays the product grid. \\ \hline
loadNextPage & - & - & Public & Loads pagination content. \\ \hline
selectLogin & - & - & Public & Navigates to Login view. \\ \hline
\end{tabular}
\end{center}

% --- V_RegistrationFormView ---
\subsubsection{Class V\_RegistrationFormView}
\noindent\textbf{Description:} The user interface responsible for capturing new user data. It presents a form for entering personal details and credentials, handling input validation feedback before submission.

\noindent\textbf{Methods:}
\begin{center}
\begin{tabular}{|p{4cm}|p{1.5cm}|p{1.5cm}|p{2cm}|p{4cm}|}
\hline
\textbf{Method} & \textbf{Input} & \textbf{Output} & \textbf{Access} & \textbf{Description} \\ \hline
showRegistrationForm & - & - & Public & Displays registration fields. \\ \hline
submitForm & - & - & Public & Submits the form data. \\ \hline
\end{tabular}
\end{center}

% --- V_CatalogPageView ---
\subsubsection{Class V\_CatalogPageView}
\noindent\textbf{Description:} The primary interface for browsing the product inventory. It displays a grid of available laptops and provides sidebar controls for filtering results by brand, price, and other specifications.

\noindent\textbf{Methods:}
\begin{center}
\begin{tabular}{|p{4cm}|p{1.5cm}|p{1.5cm}|p{2cm}|p{4cm}|}
\hline
\textbf{Method} & \textbf{Input} & \textbf{Output} & \textbf{Access} & \textbf{Description} \\ \hline
displayFilteredProducts & - & - & Public & Updates grid based on filter. \\ \hline
selectFilterOptions & - & - & Public & Captures filter selections. \\ \hline
\end{tabular}
\end{center}

% --- V_ProductPageView ---
\subsubsection{Class V\_ProductPageView}
\noindent\textbf{Description:} A detailed view dedicated to a single laptop product. It presents in-depth technical specifications, high-resolution images, and the option for the user to add the item to their shopping cart.

\noindent\textbf{Methods:}
\begin{center}
\begin{tabular}{|p{4cm}|p{1.5cm}|p{1.5cm}|p{2cm}|p{4cm}|}
\hline
\textbf{Method} & \textbf{Input} & \textbf{Output} & \textbf{Access} & \textbf{Description} \\ \hline
displayLaptopDetails & - & - & Public & Shows product specs. \\ \hline
addToCart & - & - & Public & Triggers add to cart action. \\ \hline
\end{tabular}
\end{center}

% --- V_CartPageView ---
\subsubsection{Class V\_CartPageView}
\noindent\textbf{Description:} The interface where users review their selected items. It allows users to adjust quantities, remove items, view the total cost, and proceed to the checkout process.

\noindent\textbf{Methods:}
\begin{center}
\begin{tabular}{|p{4cm}|p{1.5cm}|p{1.5cm}|p{2cm}|p{4cm}|}
\hline
\textbf{Method} & \textbf{Input} & \textbf{Output} & \textbf{Access} & \textbf{Description} \\ \hline
renderCart & - & - & Public & Lists items in the cart. \\ \hline
deleteItem & - & - & Public & Removes an item. \\ \hline
\end{tabular}
\end{center}

% --- V_PaymentPageView ---
\subsubsection{Class V\_PaymentPageView}
\noindent\textbf{Description:} The checkout interface where users select their preferred payment method. It securely collects payment details and initiates the transaction process.

\noindent\textbf{Methods:}
\begin{center}
\begin{tabular}{|p{4.2cm}|p{1.5cm}|p{1.5cm}|p{1.8cm}|p{4cm}|}
\hline
\textbf{Method} & \textbf{Input} & \textbf{Output} & \textbf{Access} & \textbf{Description} \\ \hline
displayPaymentMethods & - & - & Public & Shows payment options. \\ \hline
selectMethod & - & - & Public & Captures selected method. \\ \hline
\end{tabular}
\end{center}

% --- V_ProfilePageView ---
\subsubsection{Class V\_ProfilePageView}
\noindent\textbf{Description:} A personal dashboard for the logged-in user. It displays account details and a history of past orders, allowing users to track their purchases and manage their profile.

\noindent\textbf{Methods:}
\begin{center}
\begin{tabular}{|p{4cm}|p{1.5cm}|p{1.5cm}|p{2cm}|p{4cm}|}
\hline
\textbf{Method} & \textbf{Input} & \textbf{Output} & \textbf{Access} & \textbf{Description} \\ \hline
displayOrderHistory & - & - & Public & Shows past orders. \\ \hline
\end{tabular}
\end{center}

% --- V_OrderPageView ---
\subsubsection{Class V\_OrderPageView}
\noindent\textbf{Description:} A dedicated view for managing active and past orders. It allows users to view the status of their current orders and initiate refund requests for eligible transactions.

\noindent\textbf{Methods:}
\begin{center}
\begin{tabular}{|p{4cm}|p{1.5cm}|p{1.5cm}|p{2cm}|p{4cm}|}
\hline
\textbf{Method} & \textbf{Input} & \textbf{Output} & \textbf{Access} & \textbf{Description} \\ \hline
openRefundPopup & - & - & Public & Opens refund modal. \\ \hline
refreshMyOrders & - & - & Public & Refreshes the order list. \\ \hline
\end{tabular}
\end{center}

% --- V_ReviewPageView ---
\subsubsection{Class V\_ReviewPageView}
\noindent\textbf{Description:} The interface that allows customers to provide feedback on their purchases. It includes input controls for selecting a star rating and a text area for writing a detailed review.

\noindent\textbf{Methods:}
\begin{center}
\begin{tabular}{|p{4cm}|p{1.5cm}|p{1.5cm}|p{2cm}|p{4cm}|}
\hline
\textbf{Method} & \textbf{Input} & \textbf{Output} & \textbf{Access} & \textbf{Description} \\ \hline
pressSubmitButton & - & - & Public & Submits the review. \\ \hline
\end{tabular}
\end{center}

% --- V_DashboardPageView ---
\subsubsection{Class V\_DashboardPageView}
\noindent\textbf{Description:} The central hub for administrators. It visualizes key performance indicators (KPIs) through charts and graphs, providing a quick overview of the store's health and recent activity.

\noindent\textbf{Methods:}
\begin{center}
\begin{tabular}{|p{4cm}|p{1.5cm}|p{1.5cm}|p{2cm}|p{4cm}|}
\hline
\textbf{Method} & \textbf{Input} & \textbf{Output} & \textbf{Access} & \textbf{Description} \\ \hline
renderCharts & - & - & Public & Renders analytics charts. \\ \hline
displayDashboardError & - & - & Public & Shows admin errors. \\ \hline
\end{tabular}
\end{center}

% --- V_InventoryPageView ---
\subsubsection{Class V\_InventoryPageView}
\noindent\textbf{Description:} The administrative interface for managing the product catalog. It lists all inventory items and provides controls to add new products, update existing details, or remove obsolete items.

\noindent\textbf{Methods:}
\begin{center}
\begin{tabular}{|p{4cm}|p{1.5cm}|p{1.5cm}|p{2cm}|p{4cm}|}
\hline
\textbf{Method} & \textbf{Input} & \textbf{Output} & \textbf{Access} & \textbf{Description} \\ \hline
loadInventoryList & - & - & Public & Lists all products. \\ \hline
clickEdit & - & - & Public & Opens edit mode. \\ \hline
\end{tabular}
\end{center}

% --- V_AddProductPageView ---
\subsubsection{Class V\_AddProductPageView}
\noindent\textbf{Description:} A specialized form for administrators to input details for a new laptop product. It ensures all necessary fields like brand, price, and specs are collected before adding the item to the database.

\noindent\textbf{Methods:}
\begin{center}
\begin{tabular}{|p{4cm}|p{1.5cm}|p{1.5cm}|p{2cm}|p{4cm}|}
\hline
\textbf{Method} & \textbf{Input} & \textbf{Output} & \textbf{Access} & \textbf{Description} \\ \hline
openAddProductForm & - & - & Public & Shows the input form. \\ \hline
clickSubmit & - & - & Public & Submits the new product. \\ \hline
\end{tabular}
\end{center}

% --- V_EditProductPageView ---
\subsubsection{Class V\_EditProductPageView}
\noindent\textbf{Description:} An interface that pre-loads existing product data for modification. It allows administrators to update prices, stock levels, or specifications of a specific laptop.

\noindent\textbf{Methods:}
\begin{center}
\begin{tabular}{|p{4cm}|p{1.5cm}|p{1.5cm}|p{2cm}|p{4cm}|}
\hline
\textbf{Method} & \textbf{Input} & \textbf{Output} & \textbf{Access} & \textbf{Description} \\ \hline
openEditSession & - & - & Public & Loads product data. \\ \hline
clickUpdateLaptop & - & - & Public & Saves changes. \\ \hline
\end{tabular}
\end{center}

% --- V_OrderDashboardView ---
\subsubsection{Class V\_OrderDashboardView}
\noindent\textbf{Description:} The administrative view for overseeing all customer orders. It allows admins to filter orders by status, view details, and update the processing stage (e.g., from Pending to Shipped).

\noindent\textbf{Methods:}
\begin{center}
\begin{tabular}{|p{4cm}|p{1.5cm}|p{1.5cm}|p{2cm}|p{4cm}|}
\hline
\textbf{Method} & \textbf{Input} & \textbf{Output} & \textbf{Access} & \textbf{Description} \\ \hline
fetchOrderList & - & - & Public & Loads all orders. \\ \hline
clickUpdateStatus & - & - & Public & Updates order status. \\ \hline
\end{tabular}
\end{center}

% --- V_RefundPanelView ---
\subsubsection{Class V\_RefundPanelView}
\noindent\textbf{Description:} The administrative interface for processing refund requests. It lists pending tickets and provides controls for admins to review reasons and approve or reject the refunds.

\noindent\textbf{Methods:}
\begin{center}
\begin{tabular}{|p{4cm}|p{1.5cm}|p{1.5cm}|p{2cm}|p{4cm}|}
\hline
\textbf{Method} & \textbf{Input} & \textbf{Output} & \textbf{Access} & \textbf{Description} \\ \hline
displayPendingTickets & - & - & Public & Lists pending refunds. \\ \hline
clickSubmitDecision & - & - & Public & Approves/Rejects refund. \\ \hline
\end{tabular}
\end{center}

\subsubsection{Control Package}

% --- C_BaseController ---
\subsubsection{Class C\_BaseController}
\noindent\textbf{Description:} An abstract controller that provides shared utility functions for all other controllers. It handles common tasks such as input validation, standardized error handling, and audit logging to ensure consistency in application logic.

\noindent\textbf{Methods:}
\begin{center}
\begin{tabular}{|p{3cm}|p{2.5cm}|p{2.5cm}|p{2cm}|p{3cm}|}
\hline
\textbf{Method} & \textbf{Input} & \textbf{Output} & \textbf{Access} & \textbf{Description} \\ \hline
validateInput & payload: Map & List\textless String\textgreater & Public & Validates input. \\ \hline
handleError & error: String & void & Public & Handles errors. \\ \hline
logAudit & eventName: String \newline data: Map & void & Public & Logs events. \\ \hline
\end{tabular}
\end{center}

% --- C_RegistrationController ---
\subsubsection{Class C\_RegistrationController}
\noindent\textbf{Description:} Manages the workflow for user sign-up. It coordinates the validation of user input from the registration view and interacts with the Model to create a new user account securely.

\noindent\textbf{Methods:}
\begin{center}
\begin{tabular}{|p{3cm}|p{2.5cm}|p{2.5cm}|p{2cm}|p{3cm}|}
\hline
\textbf{Method} & \textbf{Input} & \textbf{Output} & \textbf{Access} & \textbf{Description} \\ \hline
validateFormData & userInfo: Map & List\textless String\textgreater & Public & Checks form data. \\ \hline
register & userInfo: Map & String & Public & Creates account. \\ \hline
\end{tabular}
\end{center}

% --- C_LoginController ---
\subsubsection{Class C\_LoginController}
\noindent\textbf{Description:} Handles the authentication process. It verifies user credentials against the stored records, manages session creation upon successful login, and handles the logout process to secure the session.

\noindent\textbf{Methods:}
\begin{center}
\begin{tabular}{|p{3cm}|p{3cm}|p{1.5cm}|p{2cm}|p{3.5cm}|}
\hline
\textbf{Method} & \textbf{Input} & \textbf{Output} & \textbf{Access} & \textbf{Description} \\ \hline
submitCredentials & email: String \newline pass: String & void & Public & Processes login. \\ \hline
verifyCredentials & email: String \newline pass: String & String & Public & Checks credentials. \\ \hline
logout & sessionToken: String & void & Public & Ends session. \\ \hline
\end{tabular}
\end{center}

% --- C_ProductController ---
\subsubsection{Class C\_ProductController}
\noindent\textbf{Description:} Orchestrates the retrieval and filtering of product data. It handles requests for paginated product lists, search queries, and specific product details, serving the data to the appropriate views.

\noindent\textbf{Methods:}
\begin{center}
\begin{tabular}{|p{3.5cm}|p{3cm}|p{2cm}|p{1.5cm}|p{3cm}|}
\hline
\textbf{Method} & \textbf{Input} & \textbf{Output} & \textbf{Access} & \textbf{Description} \\ \hline
fetchPaginatedProducts & page: int, size: int & List\textless Object\textgreater & Public & Gets products page. \\ \hline
fetchNextPage & cursor: String & List\textless Object\textgreater & Public & Gets next page. \\ \hline
queryProducts & keywords: String & List\textless Object\textgreater & Public & Searches products. \\ \hline
queryFilteredProducts & criteria: Map & List\textless Object\textgreater & Public & Filters products. \\ \hline
\end{tabular}
\end{center}

% --- C_CartController ---
\subsubsection{Class C\_CartController}
\noindent\textbf{Description:} Manages the state of the user's shopping cart. It handles actions such as adding items, updating quantities, and removing products, ensuring the cart's total is always accurate.

\noindent\textbf{Methods:}
\begin{center}
\begin{tabular}{|p{3cm}|p{3cm}|p{2cm}|p{1.5cm}|p{3.5cm}|}
\hline
\textbf{Method} & \textbf{Input} & \textbf{Output} & \textbf{Access} & \textbf{Description} \\ \hline
addToCart & userId: UUID \newline laptopId: UUID & Boolean & Public & Adds item to cart. \\ \hline
removeItem & itemId: UUID & Boolean & Public & Removes item. \\ \hline
queryCart & userId: UUID & Object & Public & Gets user cart. \\ \hline
\end{tabular}
\end{center}

% --- C_OrderController ---
\subsubsection{Class C\_OrderController}
\noindent\textbf{Description:} Coordinates the complex process of order creation and management. It converts a shopping cart into a finalized order, tracks its status updates, and retrieves order history for users.

\noindent\textbf{Methods:}
\begin{center}
\begin{tabular}{|p{3cm}|p{3cm}|p{2cm}|p{1.5cm}|p{3.5cm}|}
\hline
\textbf{Method} & \textbf{Input} & \textbf{Output} & \textbf{Access} & \textbf{Description} \\ \hline
placeOrder & cartInfo: Map & UUID & Public & Creates an order. \\ \hline
viewOrders & userId: UUID & List\textless Object\textgreater & Public & Lists user orders. \\ \hline
updateStatus & orderId: UUID \newline status: String & Boolean & Public & Updates status. \\ \hline
\end{tabular}
\end{center}

% --- C_PaymentGateway ---
\subsubsection{Class C\_PaymentGateway}
\noindent\textbf{Description:} An interface that abstracts the interaction with external payment processing systems. It defines the contract for creating transactions and handling success or failure responses, allowing for easy swapping of payment providers.

\noindent\textbf{Methods:}
\begin{center}
\begin{tabular}{|p{3.5cm}|p{3cm}|p{1.5cm}|p{1.5cm}|p{3.5cm}|}
\hline
\textbf{Method} & \textbf{Input} & \textbf{Output} & \textbf{Access} & \textbf{Description} \\ \hline
createTransaction & orderId: UUID & String & Public & Starts transaction. \\ \hline
transactionSuccess & txnId: String & Boolean & Public & Confirms success. \\ \hline
transactionFail & txnId: String \newline reason: String & Boolean & Public & Records failure. \\ \hline
\end{tabular}
\end{center}

% --- C_PaymentController ---
\subsubsection{Class C\_PaymentController}
\noindent\textbf{Description:} Manages the payment workflow. It uses the Payment Gateway to process transactions for specific orders and updates the order status based on the payment outcome.

\noindent\textbf{Attributes:}
\begin{center}
\begin{tabular}{|p{3cm}|p{3cm}|p{2.5cm}|p{4.5cm}|}
\hline
\textbf{Attribute} & \textbf{Datatype} & \textbf{Accessibility} & \textbf{Description} \\ \hline
gateway & C\_PaymentGateway & Private & Payment provider interface. \\ \hline
\end{tabular}
\end{center}

\noindent\textbf{Methods:}
\begin{center}
\begin{tabular}{|p{3cm}|p{3cm}|p{1.5cm}|p{1.5cm}|p{4cm}|}
\hline
\textbf{Method} & \textbf{Input} & \textbf{Output} & \textbf{Access} & \textbf{Description} \\ \hline
makePayment & orderId: UUID \newline method: String & Boolean & Public & Processes payment. \\ \hline
updateOrderStatus & orderId: UUID \newline status: String & void & Public & Syncs order. \\ \hline
\end{tabular}
\end{center}

% --- C_RefundController ---
\subsubsection{Class C\_RefundController}
\noindent\textbf{Description:} Handles the lifecycle of refund requests. It processes the submission of new refund tickets by users and coordinates the approval or rejection decisions made by administrators.

\noindent\textbf{Methods:}
\begin{center}
\begin{tabular}{|p{3.5cm}|p{3cm}|p{2cm}|p{1.5cm}|p{3cm}|}
\hline
\textbf{Method} & \textbf{Input} & \textbf{Output} & \textbf{Access} & \textbf{Description} \\ \hline
submitRefund Request & orderId: UUID \newline reason: String & UUID & Public & Creates request. \\ \hline
fetchPending Tickets & - & List\textless Object\textgreater & Public & Lists pending. \\ \hline
solveTicket & ticketId: UUID \newline decision: String \newline comments: String & Boolean & Public & Resolves ticket. \\ \hline
\end{tabular}
\end{center}

% --- C_ReviewController ---
\subsubsection{Class C\_ReviewController}
\noindent\textbf{Description:} Manages the submission and storage of product reviews. It validates the user's rating and comment before persisting the review data to the Model.

\noindent\textbf{Methods:}
\begin{center}
\begin{tabular}{|p{4cm}|p{3cm}|p{1.5cm}|p{1.5cm}|p{3cm}|}
\hline
\textbf{Method} & \textbf{Input} & \textbf{Output} & \textbf{Access} & \textbf{Description} \\ \hline
validateAndStore Review & rating: int \newline comment: String & UUID & Public & Saves review. \\ \hline
\end{tabular}
\end{center}

% --- C_AnalyticsController ---
\subsubsection{Class C\_AnalyticsController}
\noindent\textbf{Description:} Aggregates data for administrative insights. It calculates metrics such as total sales and revenue over specified date ranges to populate the admin dashboard.

\noindent\textbf{Methods:}
\begin{center}
\begin{tabular}{|p{3cm}|p{3cm}|p{1.5cm}|p{2cm}|p{3.5cm}|}
\hline
\textbf{Method} & \textbf{Input} & \textbf{Output} & \textbf{Access} & \textbf{Description} \\ \hline
getMetrics & dateRange: String & Object & Public & Calculates stats. \\ \hline
\end{tabular}
\end{center}

% --- C_InventoryController ---
\subsubsection{Class C\_InventoryController}
\noindent\textbf{Description:} Manages the CRUD (Create, Read, Update, Delete) operations for the product inventory. It handles the logic for adding new laptops, updating stock and prices, and removing products from the catalog.

\noindent\textbf{Methods:}
\begin{center}
\begin{tabular}{|p{3cm}|p{3cm}|p{1.5cm}|p{1.5cm}|p{4cm}|}
\hline
\textbf{Method} & \textbf{Input} & \textbf{Output} & \textbf{Access} & \textbf{Description} \\ \hline
createNewLaptop & specs: Map \newline price: Decimal & UUID & Public & Adds laptop. \\ \hline
deleteProduct & productId: UUID & Boolean & Public & Removes laptop. \\ \hline
updateProductInfo & laptopId: UUID \newline fields: Map & Boolean & Public & Edits laptop. \\ \hline
\end{tabular}
\end{center}

\subsection{Data Storage Design}

\subsubsection{Overview}
The database design for the Laptop Shop Management System is structured to support the system’s core retail functionalities while maintaining data integrity and scalability. The system utilizes PostgreSQL as the primary relational database management system for structured transaction and user data.

The schema is centered around a normalized relational structure where the users table serves as the primary entity for identity management. Relationships are strictly enforced through foreign keys to ensure consistency across the product catalog (laptops), sales transactions (orders), and customer feedback (reviews). This design ensures that all user actions are traceable and that product inventory is accurately managed through dedicated line items.

\begin{table}[H]
\centering
\begin{tabular}{|c|l|p{8cm}|}
\hline
\textbf{Index} & \textbf{Table Name} & \textbf{Description} \\ \hline
1 & users & Central repository for account credentials, profile information, and system roles. \\ \hline
2 & laptops & Stores extensive product details, specifications, pricing, and stock status. \\ \hline
3 & orders & Captures transaction headers, linked to users and specific shipping snapshots. \\ \hline
4 & order\_items & Links orders to specific products, recording quantities and purchase prices. \\ \hline
5 & reviews & Stores customer feedback and ratings for specific laptops. \\ \hline
6 & refund\_tickets & Manages refund requests and tracks their resolution status. \\ \hline
\end{tabular}
\caption{Database Tables Overview}
\end{table}

\subsubsection{Detailed Data Storage Design}

\textbf{1. Table: users} \\
Description: Stores user account information, authentication data, and authorization roles.

\begin{table}[H]
\centering
\begin{tabular}{|c|l|l|c|l|p{4cm}|}
\hline
\textbf{Index} & \textbf{Fieldname} & \textbf{Datatype} & \textbf{Nullable} & \textbf{Constraint} & \textbf{Description} \\ \hline
1 & id & VARCHAR(50) & No & PK, Unique & Unique identifier for the user. \\ \hline
2 & email & VARCHAR(100) & No & Unique & User's login email address. \\ \hline
3 & first\_name & VARCHAR(50) & No & & User's first name. \\ \hline
4 & last\_name & VARCHAR(50) & No & & User's last name. \\ \hline
5 & phone & VARCHAR(15) & Yes & & Contact phone number. \\ \hline
6 & created\_at & DATETIME & No & & Timestamp of account creation. \\ \hline
\end{tabular}
\caption{Users Table Design}
\end{table}

\textbf{2. Table: laptops} \\
Description: Stores extensive product details, specifications, pricing, and stock status.

\begin{table}[H]
\centering
\begin{tabular}{|c|l|l|c|l|p{4cm}|}
\hline
\textbf{Index} & \textbf{Fieldname} & \textbf{Datatype} & \textbf{Nullable} & \textbf{Constraint} & \textbf{Description} \\ \hline
1 & id & VARCHAR(10) & No & PK, Unique & Unique identifier for the laptop. \\ \hline
2 & brand & VARCHAR(50) & No & & Brand name (e.g., Dell, Asus). \\ \hline
3 & name & VARCHAR(100) & No & & Model name of the laptop. \\ \hline
4 & description & TEXT & Yes & & Detailed product description. \\ \hline
5 & cpu & VARCHAR(50) & No & & Processor specifications. \\ \hline
6 & ram\_amount & VARCHAR(20) & No & & Amount of RAM (e.g., 16GB). \\ \hline
7 & vga & VARCHAR(50) & No & & Graphics card specifications. \\ \hline
8 & quantity & INT & No & & Current stock quantity. \\ \hline
9 & original\_price & DECIMAL & No & & MSRP price. \\ \hline
10 & sale\_price & DECIMAL & No & & Actual selling price. \\ \hline
11 & product\_image & VARCHAR(255) & Yes & & URL reference to the image. \\ \hline
12 & inserted\_at & DATETIME & No & & Timestamp of record creation. \\ \hline
\end{tabular}
\caption{Laptops Table Design}
\end{table}

\textbf{3. Table: orders} \\
Description: Captures customer purchase information and transaction status.

\begin{table}[H]
\centering
\begin{tabular}{|c|l|l|c|l|p{4cm}|}
\hline
\textbf{Index} & \textbf{Fieldname} & \textbf{Datatype} & \textbf{Nullable} & \textbf{Constraint} & \textbf{Description} \\ \hline
1 & id & VARCHAR(10) & No & PK, Unique & Unique identifier for the order. \\ \hline
2 & user\_id & VARCHAR(50) & No & FK & Link to users.id. \\ \hline
3 & shipping\_address & TEXT & No & & Delivery address at time of purchase. \\ \hline
4 & phone\_number & VARCHAR(15) & No & & Contact phone for this specific order. \\ \hline
5 & payment\_method & VARCHAR(20) & No & & e.g., COD, E-banking. \\ \hline
6 & total\_price & DECIMAL & No & & Total value of the order. \\ \hline
7 & status & VARCHAR(20) & No & & Order status (Pending, Shipped, etc). \\ \hline
8 & created\_at & DATETIME & No & & Timestamp when order was placed. \\ \hline
\end{tabular}
\caption{Orders Table Design}
\end{table}

\textbf{4. Table: order\_items} \\
Description: Links orders to specific products, recording snapshots of price to ensure accurate history.

\begin{table}[H]
\centering
\begin{tabular}{|c|l|l|c|l|p{4cm}|}
\hline
\textbf{Index} & \textbf{Fieldname} & \textbf{Datatype} & \textbf{Nullable} & \textbf{Constraint} & \textbf{Description} \\ \hline
1 & id & VARCHAR(10) & No & PK, Unique & Unique identifier for the item. \\ \hline
2 & order\_id & VARCHAR(10) & No & FK & Link to orders.id. \\ \hline
3 & product\_id & VARCHAR(10) & No & FK & Link to laptops.id. \\ \hline
4 & quantity & INT & No & & Number of units purchased. \\ \hline
5 & price\_at\_purchase & DECIMAL & No & & Price at the time of transaction. \\ \hline
\end{tabular}
\caption{Order\_Items Table Design}
\end{table}

\textbf{5. Table: reviews} \\
Description: Stores customer feedback and ratings for specific laptops.

\begin{table}[H]
\centering
\begin{tabular}{|c|l|l|c|l|p{4cm}|}
\hline
\textbf{Index} & \textbf{Fieldname} & \textbf{Datatype} & \textbf{Nullable} & \textbf{Constraint} & \textbf{Description} \\ \hline
1 & id & VARCHAR(10) & No & PK, Unique & Unique identifier for the review. \\ \hline
2 & laptop\_id & VARCHAR(10) & No & FK & Link to laptops.id. \\ \hline
3 & user\_id & VARCHAR(50) & No & FK & Link to users.id. \\ \hline
4 & rating & INT & No & & Numerical rating (1-5). \\ \hline
5 & review\_text & TEXT & Yes & & Written feedback content. \\ \hline
6 & created\_at & DATETIME & No & & Timestamp of the review. \\ \hline
\end{tabular}
\caption{Reviews Table Design}
\end{table}

\textbf{6. Table: refund\_tickets} \\
Description: Manages refund requests and tracks their resolution status.

\begin{table}[H]
\centering
\begin{tabular}{|c|l|l|c|l|p{4cm}|}
\hline
\textbf{Index} & \textbf{Fieldname} & \textbf{Datatype} & \textbf{Nullable} & \textbf{Constraint} & \textbf{Description} \\ \hline
1 & id & VARCHAR(10) & No & PK, Unique & Unique identifier for the ticket. \\ \hline
2 & order\_id & VARCHAR(10) & No & FK & Link to orders.id. \\ \hline
3 & reason & TEXT & No & & Reason provided by customer. \\ \hline
4 & status & VARCHAR(20) & No & & Status (Pending, Approved, Rejected). \\ \hline
5 & created\_at & DATETIME & No & & Timestamp of request submission. \\ \hline
6 & resolved\_at & DATETIME & Yes & & Timestamp of resolution. \\ \hline
\end{tabular}
\caption{Refund\_Tickets Table Design}
\end{table}

\subsection{User Interface Design}
\subsubsection{Interface Prototype}
\subsubsection{Interface Specification}
\subsubsection{Screen Flow}

\section{Conclusion and Future Work}
\subsection{Conclusion}
\subsection{Future Work}

\section*{References}

\end{document}

